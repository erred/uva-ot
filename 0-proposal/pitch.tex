\documentclass[]{article}
\usepackage{lmodern}
\usepackage{amssymb,amsmath}
\usepackage{ifxetex,ifluatex}
\usepackage{fixltx2e} % provides \textsubscript
\ifnum 0\ifxetex 1\fi\ifluatex 1\fi=0 % if pdftex
  \usepackage[T1]{fontenc}
  \usepackage[utf8]{inputenc}
\else % if luatex or xelatex
  \ifxetex
    \usepackage{mathspec}
  \else
    \usepackage{fontspec}
  \fi
  \defaultfontfeatures{Ligatures=TeX,Scale=MatchLowercase}
\fi
% use upquote if available, for straight quotes in verbatim environments
\IfFileExists{upquote.sty}{\usepackage{upquote}}{}
% use microtype if available
\IfFileExists{microtype.sty}{%
\usepackage[]{microtype}
\UseMicrotypeSet[protrusion]{basicmath} % disable protrusion for tt fonts
}{}
\PassOptionsToPackage{hyphens}{url} % url is loaded by hyperref
\usepackage[unicode=true]{hyperref}
\hypersetup{
            pdfborder={0 0 0},
            breaklinks=true}
\urlstyle{same}  % don't use monospace font for urls
\IfFileExists{parskip.sty}{%
\usepackage{parskip}
}{% else
\setlength{\parindent}{0pt}
\setlength{\parskip}{6pt plus 2pt minus 1pt}
}
\setlength{\emergencystretch}{3em}  % prevent overfull lines
\providecommand{\tightlist}{%
  \setlength{\itemsep}{0pt}\setlength{\parskip}{0pt}}
\setcounter{secnumdepth}{0}
% Redefines (sub)paragraphs to behave more like sections
\ifx\paragraph\undefined\else
\let\oldparagraph\paragraph
\renewcommand{\paragraph}[1]{\oldparagraph{#1}\mbox{}}
\fi
\ifx\subparagraph\undefined\else
\let\oldsubparagraph\subparagraph
\renewcommand{\subparagraph}[1]{\oldsubparagraph{#1}\mbox{}}
\fi

% set default figure placement to htbp
\makeatletter
\def\fps@figure{htbp}
\makeatother


\date{}

\begin{document}

original idea: from the observation you can just add data onto the end
of a binary no elf section necessary and with suitable delimiters this
can also be applied to scripts (prefix with comment chars). this way
file signatures are always available (no separate file) and will survive
most (all?) transport methods (download over http) unlike file
attributes so it would be something like digsig/bsign

research question:

\begin{itemize}
\tightlist
\item
  is the above (hash/signature as suffix) a feasible solution to storing
  / validating signed code
\item
  does this work for compiled executables
\item
  does this work for scripts / interpreted runtimes (see link in linux
  ima)
\item
  how does this compare to existing solutions / work as an extension to
  existing framework
\end{itemize}

(incomplete) list of application whitelisting solutions by platform /
developer appname / date active

signed / integrity

\begin{itemize}
\tightlist
\item
  linux / kernel integrity measurement architecture / 2005 - present
\item
  hash based measurement
\item
  integration with TPM, measurement + attestation
\item
  uses filesystem extended attributes
\item
  possiblity of modifying interpreters to enforce ima for scripts
\item
  https://marc.info/?l=linux-kernel\&m=111682497821375\&w=2
\item
  linux / microsoft integrity policy enforcement / 2020 - present
\item
  linux security module
\item
  hash based integrity
\item
  more geared for bootup process? entire filesystem based? no docs
\item
  linux / digsig - bsign / 2002 - 2009
\item
  kernel module
\item
  enforcing signed binaries
\item
  hash / signature in extra ELF section
\item
  macos / apple gatekeeper / 10.7.3 - present
\item
  enforce signed application
\item
  windows / microsoft applocker / windows 7 - present
\item
  path, execution context, signed binary
\item
  container / docker content trust / 2015 - present
\item
  signed docker container, build / execute enforcement
\item
  cloud / google cloud binary authorization / 2018 - present
\item
  signed docker container, attestation, execute enforcement
\end{itemize}

path / other

\begin{itemize}
\tightlist
\item
  linux / redhat selinux / 2000 - present
\item
  linux security module
\item
  filesystem attribute based mandatory access control
\item
  linux / redhat fapolicyd / 2016 - present
\item
  linux security module
\item
  path based application whitelisting
\item
  integration with selinux
\item
  linux / canonical apparmor / 1998 - present
\item
  linux security module
\item
  path based mandatory access control
\item
  windows third party application whitelisting
\item
  Ivanti Application Control
\item
  McAfee Application Control
\item
  Trend Micro Application Control
\item
  Faronics Anti-Executable
\item
  Kaspersky Whitelisting
\item
  Airlock Application Whitelisting
\item
  Thycotic Application Control
\item
  \ldots{}
\end{itemize}

\end{document}
